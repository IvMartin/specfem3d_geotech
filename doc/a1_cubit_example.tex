\chapter{Building and simulation of a slope model using CUBIT/Trelis and SPECFEM3D\_GEOTECH}
\label{app:cubit}
\section*{Step 1: Create a mesh in CUBIT/Trelis}

Create a journal file "cubit\_example.jou" with following content:\\
\\
\colorbox{gray}{
\parbox{16cm}{
\noindent{\texttt{\#--------------BEGIN "cubit\_example.jou"----------------------------------------------\\
\# create model\\
create vertex 0 0 0\\
create vertex 50 0 0\\
create vertex 50 0 6\\
create vertex 43 0 6\\
create vertex 18 0 18\\
create vertex 0 0 18\\
create curve vertex 1 2\\
create curve vertex 2 3\\
create curve vertex 3 4\\
create curve vertex 4 5\\
create curve vertex 5 6\\
create curve vertex 6 1\\
create surface curve 1 2 3 4 5 6\\
sweep surface 1 vector 0 1 0 distance 20\\
compress all\\
\# mesh surface\\
surface 8 size 2\\
surface 8 scheme pave\\
mesh surface 8\\
\# mesh volume sweeping surface mesh\\
volume 1 size 2\\
volume 1 redistribute nodes off\\
volume 1 scheme Sweep source surface 8 target surface 1 sweep\_smooth Auto sweep\_transform least\_squares autosmooth\_target off\\
mesh volume 1\\
}}
}}
\newpage
% remaining part
\colorbox{gray}{
\parbox{16cm}{
\noindent{\texttt{\# define block\\
set duplicate block elements off\\
block 1 volume 1\\
\# define boundary conditions\\
Sideset 1 surface 2\\
sideset 1 name \sq{bottom\_ssbcux\_ssbcuy\_ssbcuz}\\
Sideset 2 surface 8\\
sideset 2 name \sq{front\_ssbcuy}\\
Sideset 3 surface 1\\
sideset 3 name \sq{back\_ssbcuy}\\
Sideset 4 surface 7\\
sideset 4 name \sq{left\_ssbcux}\\
Sideset 5 surface 3\\
sideset 5 name \sq{right\_ssbcux}\\
compress all\\
\# save and export mesh\\
save as "cubit\_example.cub" overwrite\\
set large exodus file off\\
export mesh "cubit\_example.e" overwrite\\
\#------------------END "cubit\_example.jou"---------------------------------------
}}
}}
\\
\\
Then open "cubit\_example.jou" file in CUBIT/Trelis and run.\\
\\
\textbf{\emph{Note}}: Above procedure can also be performed manually using CUBIT's GUI.

\section*{Step 2: Convert "Binary" EXODUS file to "ASCII" format}
\texttt{ncdump cubit\_example.e > cubit\_example.txt}\\

\textbf{\emph{Note}}: You must have installed NetCDF libraries. NetCDF is a free software which can be downloaded from
\texttt{http://www.unidata.ucar.edu/downloads/netcdf/index.jsp}

\section*{Step 3: Compile exodus2semgeotech tool}
If not already compiled, compile exodus2semgeotech.c located at util/ folder\\
\\
\texttt{gcc exodus2semgeotech.c -o exodus2semgeotech}
\\
\\
\textbf{\emph{Note}}: For more information, please check the header in exodus2semgeotech.c file.

\section*{Step 4: Convert exodus file to SPECFEM3D\_GEOTECH files}

\texttt{./exodus2sem cubit\_example.txt}\\
\\
Copy all generated files to input/ folder.

\section*{Step 5: Prepare BC files}

Open BC files "input/cubit\_example\_ssbcux", "input/cubit\_example\_ssbcuy", "input/cubit\_example\_ssbcuz". Add a line containing\\
\\
\texttt{2 0.0}\\
\\
on the top of each file and save.

\section*{Step 6: Create a material properties file}

Create a file "cubit\_example\_material\_list" in input/ folder with the following content:
\\
\\
\colorbox{gray}{
\parbox{16cm}{
\noindent{\texttt{\# material properties (id,domain,type,gamma,ym,nu,phi,coh,psi)\\
1\\
1, 1, 0, 18.8, 1e5, 0.3, 20.0, 29.0, 0.0\\
}}
}}

\section*{Step 7: Create a main input file}

Create a file "cubit\_example.sem" in input/ folder with the following content:
\\

\colorbox{gray}{
\parbox{16cm}{
\noindent{\texttt{\# -----------------BEGIN "cubit\_example.sem" ---------------------------------------------------\\
\#pre information\\
preinfo: method='sem', ngllx=3, nglly=3, ngllz=3, nenod=8, ngnod=8, \&\\
inp\_path='./input', out\_path='./output/'\\
\#mesh information\\
mesh: xfile='cubit\_example\_coord\_x', yfile='cubit\_example\_coord\_y', \&\\
zfile='cubit\_example\_coord\_z',confile='cubit\_example\_connectivity', \&\\
idfile='cubit\_example\_material\_id'\\
\#boundary conditions\\
bc: uxfile='cubit\_example\_ssbcux', uyfile='cubit\_example\_ssbcuy', \&\\
uzfile='cubit\_example\_ssbcuz'\\
\#material list\\
material: matfile='cubit\_example\_material\_list'\\
\#control parameters\\
control: cg\_tol=1e-8, cg\_maxiter=5000, nl\_tol=0.0005, nl\_maxiter=3000, \&\\
nsrf=9, srf=1.0 1.5 2.0 2.15 2.16 2.17 2.18 2.19 2.20\\
\#save data options\\
save: disp=1\\
\#-----------------END "cubit\_example.sem" ---------------------------------------------------
}}
}}

\section*{Step 8: Run prgroam}

go to SPECFEM3D\_GEOTECH folder in terminal and type\\
\\
\texttt{./bin/semgeotech ./input/cubit\_example.sem}\\
\\
\textbf{\emph{Note}}: You must have compiled SPECFEM3D\_GEOTECH. For more detail please see Chapter~\ref{chap:2}.

\section*{Step 9: Visualize result}

Start ParaView and Open "output/cubit\_example.case" file\\
\\
\textbf{\emph{Note}}: You must have installed ParaView. ParaView is a free opensource visualization software which
can be downloaded from \texttt{http://www.paraview.org/}.

